% ------------------------------------------------------------
\begin{frame}{Modeling and Interpreting}
\vfill
\begin{center}{%
\begin{picture}(300,120)(-140,-60)
\put(-80,+40){\makebox(0,0){\framebox(80,20){Problem}}}
\put(-80,-40){\makebox(0,0){\framebox(80,20){Logic Program}}}
\put(+80,+40){\makebox(0,0){\framebox(80,20){Solution}}}
\put(+80,-40){\makebox(0,0){\framebox(80,20){Stable Models}}}
\put(-80,+30){\vector(0,-1){60}}
\put(-40,-40){\vector(+1,0){80}}
\put(+80,-30){\vector(0,+1){60}}
\put(-110,  0){\makebox(0,0){\alert<2>{Modeling}}}
\put(+120,  0){\makebox(0,0){\alert<2>{Interpreting}}}
\put(   0,-55){\makebox(0,0){{Solving}}}
\end{picture}}
\end{center}
\end{frame}
%------------------------------------------------------------
\begin{frame}{Problems as Logic Program}

\bigskip{}
For solving a problem class \alert{P} for a problem instance \alert{I},

encode
\begin{enumerate}
\item the problem instance \alert{I} as a set \alert{C(I)} of facts
 and
\item the problem class    \alert{P} as a set \alert{C(P)} of rules
\end{enumerate}
such that the solutions to \alert{P} for \alert{I} can be (polynomially) extracted

from the stable models of \alert{C(I)}${}\cup{}$\alert{C(P)}.

\pause
\bigskip
\begin{block}{Uniform encoding}
A \alert{uniform} encoding \alert{C(P)} is a first-order logic program,

encoding the solutions to \alert{P} for any set \alert{C(I)} of facts.
\end{block}
\end{frame}
%------------------------------------------------------------
\subsection{Graph Coloring}
%------------------------------------------------------------
\begin{frame}{$N$-Colorability\\
  \normalsize Problem \alert{Instance} \onslide<3->{as \alert<3->{Facts}}}

%  \bigskip{}
  \begin{block}{\textbf{Given:} a (directed) \alert{graph $G$}}
%  \pause%
%  \begin{tabular}{l@{~}l}
\alt<1-2>{%
%\vspace*{-5mm}
  \begin{minipage}[t]{0.5\linewidth}
  \vspace*{1mm}
  \(
  G =
  \left(\begin{array}{@{}r@{}l@{}}
   V = \{&1,2,3,4,5,6\},
  {} \\[1mm]
   E = \{&(1,2),(1,3),(1,4),
  {} \\
         &(2,4),(2,5),(2,6),
  {} \\
         &(3,1),(3,4),(3,5),
  {} \\
         &(4,1),(4,2),
  {} \\
         &(5,3),(5,4),(5,6),
  {} \\
         &(6,2),(6,3),(6,5)\}
  \end{array}
  \right)
  \)
%\vspace*{10mm}
  \end{minipage}
}{%
\onslide<4->{%
%\vspace*{-22mm}
  \begin{minipage}[t]{0.5\linewidth}
  \vspace*{1mm}
  \begin{semiverbatim}
  \alert{node}(1). \ \alert{node}(2). \ \alert{node}(3).
  \alert{node}(4). \ \alert{node}(5). \ \alert{node}(6).
  \end{semiverbatim}

  \small
  \begin{semiverbatim}
  \alert{edge}(1,2). \alert{edge}(1,3). \alert{edge}(1,4).
  \alert{edge}(2,4). \alert{edge}(2,5). \alert{edge}(2,6).
  \alert{edge}(3,1). \alert{edge}(3,4). \alert{edge}(3,5).
  \alert{edge}(4,1). \alert{edge}(4,2).
  \alert{edge}(5,3). \alert{edge}(5,4). \alert{edge}(5,6).
  \alert{edge}(6,2). \alert{edge}(6,3). \alert{edge}(6,5).
  \end{semiverbatim}
%\vspace*{4mm}
  \end{minipage}
}}
%  & \pause%
  \onslide<2->{%
  \begin{minipage}[t]{0.45\linewidth}
  \input{Figures/graph}
  \end{minipage}}
%  \end{tabular}
  \end{block}

  % \begin{description}
  % \item[Problem instance] \ \\ A graph $(V,E)$.
  % \item[Problem class] \ \\ Assign each vertex in $V$ one of $n$ colors such that no
  %   two vertexes in $V$ connected by an edge in $E$ have the same color.
  % \end{description}
\end{frame}
%------------------------------------------------------------
\begin{frame}{$N$-Colorability\\
              \normalsize \onslide<7->{\alert<8>{(Extended)}} Problem \alert{Encoding}}
\begin{minipage}[t]{0.47\linewidth}
\begin{block}{Natural Language}
  \begin{enumerate}
  \item \alert<1-2,7-8>{Each node has
        a} \alert<1,3,7-8>{unique} \alert<1-2,7-8>{color}.
  \item[]
  \item[]
%  \item[]
\vspace{4mm}
  \item<4-> \alert<4>{Any two connected nodes must not have
                  the same color.}
%  \item[]
  \item<5-> \alert<5>{Let there be three colors.}
\vspace{3mm}
  \item<6-> \alert<6>{A solution is a coloring.}
  \end{enumerate}
\end{block}
\end{minipage}
\begin{minipage}[t]{0.47\linewidth}
\begin{block}{Logical Language}
  \begin{enumerate}
  \item<2->\small
\alt<1-7>{%
\begin{semiverbatim}
\alert<2,7>{color(X,C) :-} \alert<7>{iscol(C),} \alert<2,7>{node(X), not other(X,C).}
\end{semiverbatim}
\onslide<3-7>{%
\begin{semiverbatim}
\alert<3,7>{other(X,C) :-} \alert<7>{iscol(C),} \alert<3,7>{color(X,D), D != C.}
\end{semiverbatim}}}
{%
\onslide<8->{%
\begin{semiverbatim}
\alert<8>{1 \#count\{\ color(X,C) :

          \ \ \ \ \ \ \ \ \ \ \ \ iscol(C) \}\ 1

          \ \ \ \ \ \ \ \ :- node(X).}
\end{semiverbatim}}
\begin{semiverbatim}

\end{semiverbatim}}
\vspace{-1mm}
  \item<4->\small
\begin{semiverbatim}
\alert<4>{:- color(X,C), color(Y,C),}
\alert<4>{\ \ \ edge(X,Y).}
\end{semiverbatim}
\vspace{1mm}
% \item[]
  \item<5->\small
\begin{semiverbatim}
\#const \alert<5>{n=3}.

\alert<5>{iscol(1..n).}
\end{semiverbatim}
  \item<6->\small
\begin{semiverbatim}
\alert<6>{\#show color/2.}
\end{semiverbatim}
  \end{enumerate}
\end{block}
\end{minipage}
\end{frame}
%------------------------------------------------------------
\begin{frame}{$N$-Colorability\\
              \normalsize Recapitulation I\phantom{I}}
\begin{block}{Instance as Facts (in \attach{Examples/graph.lp}{\lstinline{graph.lp}})}
% \scriptsize
\lstinputlisting{Examples/graph.lp}
\end{block}
\end{frame}
%------------------------------------------------------------
\begin{frame}{$N$-Colorability\\
              \normalsize Recapitulation II}
\begin{block}{Uniform Encoding (in \attach{Examples/color.lp}{\lstinline{color.lp}})}
\lstinputlisting{Examples/color.lp}
\end{block}
\end{frame}
%------------------------------------------------------------
\begin{frame}[fragile]{$N$-Colorability\\
              \normalsize Let's \alert{Run} it!}
\begin{block}{\alert<1>{\lstinline{gringo} \attach{Examples/graph.lp}{\lstinline{graph.lp}} \attach{Examples/color.lp}{\lstinline{color.lp}} \lstinline{| clasp 0}}}
\vspace*{-4mm}
\pause\footnotesize%
\begin{semiverbatim}
clasp version 2.0.2
Reading from stdin
Solving...
Answer: 1
color(6,2) color(5,3) color(4,2) color(3,1) color(2,1) color(1,3)
Answer: 2
color(6,1) color(5,3) color(4,1) color(3,2) color(2,2) color(1,3)
Answer: 3
color(6,3) color(5,2) color(4,3) color(3,1) color(2,1) color(1,2)
Answer: 4
color(6,1) color(5,2) color(4,1) color(3,3) color(2,3) color(1,2)
Answer: 5
color(6,3) color(5,1) color(4,3) color(3,2) color(2,2) color(1,1)
Answer: 6
color(6,2) color(5,1) color(4,2) color(3,3) color(2,3) color(1,1)
\end{semiverbatim}
\end{block}
\end{frame}
%------------------------------------------------------------
\begin{frame}[fragile]{$N$-Colorability\\
              \normalsize Let's \alert{Interpret} it!}
\begin{block}{{\textbf{Found:} $3$-coloring(s)}}
\begin{minipage}[t]{0.48\linewidth}
\begin{semiverbatim}
Answer: 1

\alert<2>{color(1,3) color(5,3)}

\alert<3>{color(2,1) color(3,1)}

\alert<4>{color(4,2) color(6,2)}
\end{semiverbatim}
\end{minipage}
\begin{minipage}[t]{0.45\linewidth}
\input{figures/color}
\end{minipage}
\end{block}
\end{frame}
%------------------------------------------------------------
\begin{frame}[fragile]{Interlude: Stable Model(s) Computation}
\begin{center}
\begin{picture}(300,85)(-115,-40)
%\unitlength0.85pt
\thicklines
% \put(-90,+40){\makebox(0,0){\framebox(80,20){Problem}}}
% \put(-90,-40){\makebox(0,0){\framebox(80,20){Logic Program}}}
% \put(+90,+40){\makebox(0,0){\framebox(80,20){Solution}}}
% \put(+90,-40){\makebox(0,0){\framebox(80,20){stable model(s)}}}
% \put(-90,+30){\vector(0,-1){59.5}}
\put(-80,22.5){\makebox(0,0){\framebox(80,30){\begin{tabular}{c}Problem\\Instance\end{tabular}}}}
\put(-80,-102.5){\makebox(0,0){\framebox(80,30){\begin{tabular}{c}Problem\\Encoding\end{tabular}}}}
\put(-80,-40){\oval(50,20)}\put(-102,-44){Grounder}
\put(57.5,-40){\makebox(0,0){\framebox(80,30){\begin{tabular}{c}Propositional\\Logic Program\end{tabular}}}}
\put(155,-40){\oval(50,20)}\put(141,-44){Solver}
\put(155,17.5){\makebox(0,0){\framebox(80,20){Stable Models}}}
\put(+155,-30){\vector(0,+1){37}}
\put(-54.75,-40){\vector(+1,0){72}}
\put(98,-40){\vector(+1,0){32}}
\put(-80,-87.5){\vector(0,+1){37}}
\put(-80,7){\vector(0,-1){37}}
\put(-25,-30){\makebox(0,0){\alert<2>{Grounding}}}
\put(133,-12){\makebox(0,0){\alert<2>{Solving}}}
% \put(+130,  0){\makebox(0,0){{Interpreting}}}
% \put(   0,-52){\makebox(0,0){{Solving}}}
\end{picture}
\end{center}
\end{frame}
%------------------------------------------------------------
\begin{frame}[fragile]{$N$-Colorability\\
              \normalsize Grounding}
\begin{block}{\alert<1>{\lstinline{gringo} -t \attach{Examples/graph.lp}{\lstinline{graph.lp}} \attach{Examples/color.lp}{\lstinline{color.lp}}}}
\vspace*{-4mm}
\pause\footnotesize%
\begin{semiverbatim}
node(1). \ \ node(2). \ \ node(3). \ \ node(4). \ \ node(5). \ \ node(6).
edge(1,2). edge(1,3). edge(1,4). edge(2,4). edge(2,5). \dots

iscol(1). \ iscol(2). \ iscol(3).

1 \#count\{ color(1,1), color(1,2), color(1,3) \} 1.
1 \#count\{ color(2,1), color(2,2), color(2,3) \} 1.
1 \#count\{ color(3,1), color(3,2), color(3,3) \} 1.
1 \#count\{ color(4,1), color(4,2), color(4,3) \} 1.
1 \#count\{ color(5,1), color(5,2), color(5,3) \} 1.
1 \#count\{ color(6,1), color(6,2), color(6,3) \} 1.

:- color(1,1), color(2,1).
:- color(1,2), color(2,2).
:- color(1,3), color(2,3). \dots
\end{semiverbatim}
\end{block}
\end{frame}
%------------------------------------------------------------
\begin{frame}[fragile]{$N$-Colorability\\
              \normalsize Solving}
\begin{block}{\alert<1>{\lstinline{gringo} \attach{Examples/graph.lp}{\lstinline{graph.lp}} \attach{Examples/color.lp}{\lstinline{color.lp}} \lstinline{| clasp --stats 0}}}
\vspace*{-4mm}
\pause\scriptsize%
\begin{semiverbatim}
\dots{}
\alert<3>{Models      : 6}
\alert<3>{Time        : 0.001s} (Solving: 0.00s 1st Model: 0.00s Unsat: 0.00s)
CPU Time    : 0.000s
\alert<3>{Choices     : 5}
\alert<3>{Conflicts   : 0}
\alert<3>{Restarts    : 0}

Atoms       : 63
Rules       : 113    (1: 95 2: 12 3: 6)
Bodies      : 64
Equivalences: 106    (Atom=Atom: 31 Body=Body: 6 Other: 69)
\alert<3>{Tight       : Yes}

\alert<3>{Variables   : 63}     (Eliminated: 0 Frozen: \ \ 30)
\alert<3>{Constraints : 45}     (Binary: 73.3% Ternary:  0.0% Other: 26.7%)
Lemmas      : 0      (Binary:  0.0% Ternary:  0.0% Other:  0.0%)
\end{semiverbatim}\pause
\end{block}
\end{frame}
%------------------------------------------------------------
\subsection{Hamiltonian Cycle}
%------------------------------------------------------------
\begin{frame}{Hamiltonian Cycle\\
  \normalsize Problem Instance as Facts}

  \begin{block}{\textbf{Recall:} a directed graph $G$}
  \begin{minipage}[t]{0.5\linewidth}
  \vspace*{1mm}
  \begin{semiverbatim}
  node(1). \ node(2). \ node(3).
  node(4). \ node(5). \ node(6).
  \end{semiverbatim}

  \small
  \begin{semiverbatim}
  edge(1,2). edge(1,3). edge(1,4).
  edge(2,4). edge(2,5). edge(2,6).
  edge(3,1). edge(3,4). edge(3,5).
  edge(4,1). edge(4,2).
  edge(5,3). edge(5,4). edge(5,6).
  edge(6,2). edge(6,3). edge(6,5).
  \end{semiverbatim}
  \end{minipage}
  \small{\,}
  \begin{minipage}[t]{0.45\linewidth}
  \input{Figures/graph}
  \end{minipage}
  \end{block}
\end{frame}
%------------------------------------------------------------
\begin{frame}{Hamiltonian Cycle\\
  \normalsize Engineering an Encoding}
  \begin{block}{Problem Specification}
  A (directed) graph $G=(V,E)$ is Hamiltonian if it contains

  \alert<1>{a cycle~$C$ that visits every node of~$V$ exactly once}.
%  such that
  % \begin{enumerate}
  % \item a permutation $(v_1,\dots,v_i,v_{i+1},\dots,v_n)$ of $V$ such that
  % \item $\{(v_1,v_2),\dots,(v_i,v_{i+1}),\dots,(v_n,v_1)\}\subseteq E$.
  % \end{enumerate}
  \end{block}
\pause
  \begin{itemize}
  \item[\ithand] $C$ traverses \alert<3-4>{exactly} \alert<3>{one incoming} and \alert<4>{one outgoing} \alert<3-4>{edge per node}.
  \item[\ithand] $C$ \alert<5>{traverses} \alert<8>{every node} of~$V$ (\alert<7>{starting from an arbitrary node} in $V$).
  \end{itemize}
\pause
  \begin{block}{Problem Encoding}
  \begin{semiverbatim}
  \alt<1-4>{%
  \onslide<3-4>{\alert<3>{1 \#count\{\ cycle(X,Y) : edge(X,Y) \}\ 1 :- node(Y).}}

  \onslide<4>{\alert<4>{1 \#count\{\ cycle(X,Y) : edge(X,Y) \}\ 1 :- node(X).}}
}{%
  \only<5->{\alert<5>{reach(X) :- first(X).}}

  \only<5->{\alert<5>{\alert<6>{reach(Y)} :- \alert<6>{reach(X)}, cycle(X,Y).}}
}
  \end{semiverbatim}
\only<6>{%
\vspace{-2mm}
  \begin{itemize}
  \item[\ithand] The definition of \lstinline{reach} is \alert{recursive}!
  \end{itemize}
}
  \begin{semiverbatim}
  \alt<1-7>{\only<7>{\alert<7>{first(X) :- X = \#min[ node(Y) = Y ].}}}{%
            \onslide<8->{\alert<8>{:- node(Y), not reach(Y).}}}
  \end{semiverbatim}
  \end{block}
\end{frame}
%------------------------------------------------------------
\begin{frame}{Hamiltonian Cycle\\
              \normalsize The Complete Picture}
\begin{block}{Uniform Encoding (in \attach{Examples/cycle.lp}{\lstinline{cycle.lp}})}
\lstinputlisting[basicstyle=\ttfamily\scriptsize]{Examples/cycle.lp}
\end{block}
\end{frame}
%------------------------------------------------------------
\begin{frame}[fragile]{Hamiltonian Cycle\\
              \normalsize Let's \alert{Run} it!}
\begin{block}{\alert<1>{\lstinline{gringo} \attach{Examples/graph.lp}{\lstinline{graph.lp}} \attach{Examples/cycle.lp}{\lstinline{cycle.lp}} \lstinline{| clasp --stats}}}
\vspace*{-2mm}
\pause\scriptsize%
\begin{semiverbatim}
Answer: 1
cycle(6,5) cycle(5,3) cycle(4,2) cycle(3,1) cycle(2,6) cycle(1,4)
SATISFIABLE

Models      : 1+
Time        : 0.001s (Solving: 0.00s 1st Model: 0.00s Unsat: 0.00s)
CPU Time    : 0.000s
Choices     : 3
Conflicts   : 0
Restarts    : 0

Atoms       : 84
Rules       : 117    (1: 84 2: 21 3: 12)
Bodies      : 81
Equivalences: 174    (Atom=Atom: 36 Body=Body: 12 Other: 126)
\alert{Tight       : No     (SCCs: 1 Nodes: 20)}
\end{semiverbatim}
\end{block}
\end{frame}
%------------------------------------------------------------
\begin{frame}[fragile]{Hamiltonian Cycle\\
              \normalsize Let's \alert{Interpret} it!}
\begin{block}{{\textbf{Found:} Hamiltonian cycle}}
\begin{minipage}[t]{0.48\linewidth}
\begin{semiverbatim}
Answer: 1

\alert<2>{cycle(1,4)}
\alert<3>{cycle(4,2)}
\alert<4>{cycle(2,6)}
\alert<5>{cycle(6,5)}
\alert<6>{cycle(5,3)}
\alert<7>{cycle(3,1)}
\end{semiverbatim}
\end{minipage}
\begin{minipage}[t]{0.45\linewidth}
\input{figures/cycle}
\end{minipage}
\end{block}
\end{frame}
%------------------------------------------------------------
\subsection{Traveling Salesperson}
%------------------------------------------------------------
\begin{frame}{Mr Hamilton as Traveling Salesperson\\
  \normalsize Problem Instance as Facts}
  \begin{block}{\textbf{Given:} a directed graph $G$ \onslide<2->{\alert{plus edge costs}}}
  \begin{minipage}[t]{0.5\linewidth}
  \vspace*{1mm}
\alt<1>{%
  \begin{semiverbatim}
  node(1). \ node(2). \ node(3).
  node(4). \ node(5). \ node(6).
  \end{semiverbatim}

  \small
  \begin{semiverbatim}
  edge(1,2). edge(1,3). edge(1,4).
  edge(2,4). edge(2,5). edge(2,6).
  edge(3,1). edge(3,4). edge(3,5).
  edge(4,1). edge(4,2).
  edge(5,3). edge(5,4). edge(5,6).
  edge(6,2). edge(6,3). edge(6,5).
  \end{semiverbatim}
}{%
  \small
  \begin{semiverbatim}
  \alert{cost}(1,2,2).

  \alert{cost}(1,3,3). \alert{cost}(3,1,3).

  \alert{cost}(1,4,1). \alert{cost}(4,1,1).

  \alert{cost}(2,4,2). \alert{cost}(4,2,2).

  \alert{cost}(2,5,2).

  \alert{cost}(2,6,4). \alert{cost}(6,2,4).

  \alert{cost}(3,4,2).

  \alert{cost}(3,5,2). \alert{cost}(5,3,2).

  \alert{cost}(5,4,2).

  \alert{cost}(5,6,1). \alert{cost}(6,5,1).

  \alert{cost}(6,3,3).
  \end{semiverbatim}
}
  \end{minipage}
  \small{\,}
  \begin{minipage}[t]{0.45\linewidth}
  \input{Figures/costs}
  \end{minipage}
  \end{block}
\end{frame}
%------------------------------------------------------------
\begin{frame}{Mr Hamilton as Traveling Salesperson\\
  \normalsize Solution \alert{Optimization}}
  \begin{block}{Optimization Objective}
  A Hamiltonian cycle is optimal if
  its accumulated edge costs are \alert<1>{minimal}.
  \end{block}
  \begin{itemize}
  \item[\ithand] Use \alert<1>{\lstinline{\#minimize}} (and/or \alert<1>{\lstinline{\#maximize}})
                 to associate each stable model with objective value(s).
  \end{itemize}
\pause
  \begin{block}{Optimization Encoding}
  \vspace*{-2mm}
  \begin{semiverbatim}
  \% OPTIMIZE

  \alert<2>{\#minimize}[ \alert<6>{cycle(X,Y)} \alert<5>{: cost(X,Y,C)} \alert<3>{= C}\alert<4>{@1} ].
  \end{semiverbatim}
  \vspace{-2mm}
  \begin{description}
  \item[Target:] \onslide<2->{\alert<2>{minimal sum}}
                 \onslide<3->{\alert<3>{of costs \lstinline{C}}}
                 \onslide<4->{\alert<4>{(at priority level~\lstinline{1})}}
                 \onslide<5->{\alert<5>{associated with}}
                 \onslide<6->{\alert<6>{instances of \lstinline{cycle}
                 in an stable model}}
  \end{description}
  \end{block}
\end{frame}
%------------------------------------------------------------
\begin{frame}[fragile]{Mr Hamilton as Traveling Salesperson\\
              \normalsize Let's \alert{Run} it!}
\begin{block}{\alert<1>{\lstinline{gringo} \attach{Examples/graph.lp}{\lstinline{graph.lp}} \attach{Examples/costs.lp}{\lstinline{costs.lp}} \attach{Examples/cycle.lp}{\lstinline{cycle.lp}} \attach{Examples/price.lp}{\lstinline{price.lp}} \lstinline{| clasp --stats 0}}}
\vspace*{-4mm}
\pause\scriptsize%
\begin{semiverbatim}
Answer: 1
cycle(6,5) cycle(5,3) cycle(4,2) cycle(3,1) cycle(2,6) cycle(1,4)
\alert{Optimization: 13}
Answer: 2
cycle(6,5) cycle(5,3) cycle(4,1) cycle(3,4) cycle(2,6) cycle(1,2)
\alert{Optimization: 12}
Answer: 3
cycle(6,3) cycle(5,6) cycle(4,1) cycle(3,4) cycle(2,5) cycle(1,2)
\alert{Optimization: 11}
\alert{OPTIMUM FOUND}

\alert{Models      : 1}
\ \ \alert{Enumerated: 3}
\ \ \alert{Optimum   : yes}
\alert{Optimization: 11}
Time        : 0.004s (Solving: 0.00s 1st Model: 0.00s Unsat: 0.00s)
CPU Time    : 0.000s
\end{semiverbatim}
\end{block}
\end{frame}
%------------------------------------------------------------
\begin{frame}[fragile]{Mr Hamilton as Traveling Salesperson\\
              \normalsize Let's \alert{Interpret} it!}
\begin{block}{\textbf{Found:} \onslide<3->{\alert{optimal}} Hamiltonian cycle}
\begin{minipage}[t]{0.48\linewidth}
\begin{semiverbatim}
Answer: \alt<1>{1

cycle(1,4)
cycle(4,2)
cycle(2,6)
cycle(6,5)
cycle(5,3)
cycle(3,1)}{\alt<2>{2

cycle(1,2)
cycle(2,6)
cycle(6,5)
cycle(5,3)
cycle(3,4)
cycle(4,1)}{3

cycle(1,2)
cycle(2,5)
cycle(5,6)
cycle(6,3)
cycle(3,4)
cycle(4,1)}}
\end{semiverbatim}
\end{minipage}
\begin{minipage}[t]{0.45\linewidth}
\input{figures/price}
\end{minipage}
\end{block}
\end{frame}
%------------------------------------------------------------
\begin{frame}{Take-Home Messages}

\bigskip{}
For solving a problem (class) in ASP, provide
\begin{enumerate}
\item<1-9,14> \alert{facts} describing an instance and
\item<1-8,14> a (uniform) \alert{encoding} of solutions.
\end{enumerate}

\bigskip{}
Encodings are often structured by the following logical parts:
\begin{enumerate}
\item<2-9,14> \alert{Domain} information \hfill (by deduction from facts)
\item<3-8,10,14> \alert{Generator} providing solution candidates \hfill (choice rules)
\item<4-8,10,14> \alert{Define} rules analyzing properties of candidates \hfill (normal rules)
\item<5-8,11,14> \alert{Tester} eliminating invalid candidates \hfill (integrity constraints)
\item<6-8,12,14> \alert{Display} statements projecting stable models \hfill (onto characteristic atoms)
\item<7-8,13,14> \alert{Optimizer} evaluating stable models \hfill (\lstinline{\#minimize}/\lstinline{\#maximize})
\end{enumerate}
\onslide<8->{
\begin{block}{In a Nutshell}
\begin{tabular}{rcl}
  Logic Program&$\subseteq$&\alert<9>{(Data + Deduction)} + \alert<10>{(Generation + Analysis)} + \\
               & &\alert<11>{Selection} + \alert<12>{Projection} [+ \alert<13>{Optimization}]
\end{tabular}
\end{block}}
\end{frame}
%%% Local Variables:
%%% mode: latex
%%% TeX-PDF-mode: t
%%% TeX-master: "asp"
%%% End:
