% --------------------------------------------------------------------------------
\begin{frame}{Encode with care}
  \begin{enumerate}
  \item Create a \alert<1>{working} encoding
    \begin{enumerate}
    \item<2-> Do you need to modify the encoding if the facts change?
    \item<2-> Are all variables significant (or statically functionally dependent)?
    \item<2-> Can there be (many) identic ground rules?
    \item<2-> Do you enumerate pairs of values (to test uniqueness)?
    \item<2-> Do you assign dynamic aggregate values (to check a fixed bound)?
    \item<2-> Do you admit (obvious) symmetric solutions?
    \item<3-> Do you have additional domain knowledge simplifying the problem?
    \item<4-> Are you aware of anything else that, if encoded, would reduce grounding and/or solving efforts?
    \end{enumerate}
    \smallskip
  \item<5-> Revise until no ``Yes'' answer
    \smallskip
  \item<6->[] \structure{Note} \
    If the format of facts makes encoding painful (eg abusing grounding for ``calculations''), revise it
  \end{enumerate}
\end{frame}
% --------------------------------------------------------------------------------
\begin{frame}{Some hints on (preventing) debugging}
  \begin{itemize}
  \item \structure{Kinds of errors}
    \begin{itemize}
    \item syntactic \hfill\onslide<2->{\dots{} follow error messages by the grounder}
    \item semantic  \hfill\onslide<3->{\dots{} (most likely) encoding/intention mismatch}
    \end{itemize}
    \bigskip
  \item<4-> \structure{Ways to identify semantic errors (early)}
    \begin{itemize}
    \item<4-> develop and \alert<4>{test} incrementally
      \begin{itemize}
      \item prepare toy instances with ``interesting features''
      \item build the encoding bottom-up and verify additions (eg new predicates)
      \end{itemize}
      \smallskip
    \item<5-> compare the encoded to the \alert<5>{intended meaning}
      \begin{itemize}
      \item check whether the grounding fits (use \alert<5>{\lstinline{gringo --text}})
      \item if stable models are unintended, investigate conditions that fail to hold
      \item if stable models are missing, examine integrity constraints (add heads)
      \end{itemize}
      \smallskip
    \item<6-> ask tools\only<7>{\footnote{\url{http://www.kr.tuwien.ac.at/research/projects/mmdasp/}}}
  \end{itemize}
\end{itemize}
\end{frame}
%------------------------------------------------------------
\begin{frame}{Overcoming performance bottlenecks}
  \begin{itemize}
  \item \structure{Grounding}
    \smallskip
    \begin{itemize}
    \item monitor \alert<1-2>{time} spent by and output \alert<1-2>{size} of \lstinline{gringo}
      \begin{itemize}
      \item system tools   \ (eg.\ \lstinline{time(gringo [}\dots{}\lstinline{] | wc)})
      \item grounding info \ (eg.\ \lstinline{gringo --text})
      \end{itemize}
    \item<2-> once identified, \alert<1-2>{reformulate} ``critical'' logic program parts
    \end{itemize}
    \medskip
  \item<3-> \structure{Solving}
    \smallskip
    \begin{itemize}
    \item check solving statistics (using \lstinline{clasp --stats})
      \smallskip
    \item<4-> if great search efforts (\lstinline{conflicts}/\lstinline{choices}/\lstinline{restarts}), then
      \smallskip
      \begin{enumerate}
      \item<5-> try prefabricated settings (using \lstinline{clasp} option `\lstinline{--configuration}')
      \item<5-> try auto-configuration (offered by \lstinline{claspfolio} or \lstinline{accclingo})
      \item<5-> try manual fine-tuning (requires expert knowledge!)
      \item<6-> if possible, reformulate the problem or add domain knowledge
        (``redundant'' constraints) to help the solver
      \end{enumerate}
    \end{itemize}
  \end{itemize}
\end{frame}
%------------------------------------------------------------
%
%%% Local Variables:
%%% mode: latex
%%% TeX-master: "../../main"
%%% End:
